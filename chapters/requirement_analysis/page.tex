\chapter{Requirement Analysis}\label{chp:requirement_analysis}
\textit{"What is a \textbf{requirement}"}\\
Something the product must do or a quality that the product must have.\\
Functional requirement, what the system must do.\\
Non-functional requirement, qualities that the system must have. Characteristics and/or parameters. Think qualitative.\\\\
\textbf{Requirements needs to be testable}
\\\\
\begin{itemize}
    \item What people do
    \item What people want to do
    \item How people do what they do
\end{itemize}

\section{Terms used for activity}
Requirement gathering, pre-existing specifications:
\begin{itemize}
    \item Laws/legislatsion
    \item Documents from customer/user
\end{itemize}
Requirement generation, build from various sources.\\
Requirement solicitation, going deeper than user request.\\
Requirement engineering, active role, building the requirements.

\begin{figure}[H]
    \begin{center}
        \includegraphics*[width=\linewidth*3/4]{chapters/requirement_analysis/figures/requirement_table.png}        
    \end{center}
    \caption{Requirements table}
    \label{fig:requirement_table}
\end{figure}

Prioritizing requirements, can be done using MoSCoW model.